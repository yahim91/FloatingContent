\chapter{Mobile social networking}\label{chap1}
\thispagestyle{plain}

% ====SECTION 1 ================================================================
\section{Introduction}

Nowadays mobile devices have become a staple of our society, with everyone of
us owning at least one. They make our lives easier by giving us a wide range of
features from the most usual like texting a friend, to the most recent ones like
watching a live video of a friend. We are now able to be in contact with
everyone no matter where we are. It eliminated distances between us and
freed us from the constraints of space giving us the opportunity to communicate
with each other regardless of the location.
At the heart of mobile devices are mobile apps on which we are increasingly
relying on them for various activities. They enable us to create, share and
exchange information and ideas in virtual networks and communities.

\section{Social mobile applications}

Increasing mobile Internet use has made information sharing experiences very
popular among the users. There are many type of services on which the content is
being shared. The most popular is the social network Facebook, which enables you
to share photos and personal content with your friends. Another one is Twitter
which enables you to share information in something similar with a blog.
The evolution of location-aware mobile technology has influenced the mobile
application industry offering users a more contextual experience. The Facebook
Messenger provides the users with the location of their communication partner
and the latest feature of Nearby Friends notifies the user whether a friend is
in the nearby location. The best examples are Google Maps and Google Earth whose
purpose is to store data and display the geographic proximity based on the
position of the user.

\section{Weaknesses of network-based mobile application}
